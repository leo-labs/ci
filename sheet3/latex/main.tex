% compile this on sharelatex.com
% !TEX program = pdflatex

\documentclass{scrartcl}
\usepackage[utf8]{inputenc}
\usepackage{graphicx}
\usepackage{float}
\usepackage{dsfont}


\graphicspath{ {img/} }

\title{Submission Sheet 3}
\author{Danny Heinrich \and Matthias Kasperidus \and Leonard Kleinhans}
\date{26. November 2014}



\usepackage{amsmath}

\newcommand*\colvec[2]{
        \begin{pmatrix} #1 \\ #2 \end{pmatrix}
}

\begin{document}
\maketitle

\section{Exercise 3.1: Fuzzy Sets}

\section{Exercise 3.2: Fuzzy Complements}
As defined in lecture a fuzzy complement obtained from an arbitrary increasing generator is a function $c: [0,1] \to [0,1]$ where
$\forall a \in [0,1]: c(a) = g^{(-1)}\left(g(1)-g(a)\right)$ and $\exists$ a continuous function $g:[0,1] \to \mathds{R}$ with $g(0) = 0$ and $g$ strictly monotone increasing. \\
\\

\textbb{To be proven:} $\forall a \in [0,1]: c(c(a)) = a$ \\
Because $g(0) = 0$, $g$ continuous and $g$ strictly monotone increasing, $g$ is bijective in [0,1] and the inverse function exists (in [0,1]).
\begin{align*}
        c(c(a)) &= c \left( g^{(-1)} \left( g(1) - g(a) \right) \right) \\
                &= g^{(-1)} \left( g(1) - g \left( g^{(-1)} \left( g(1) - g(a) \right) \right) \right) \\
                &= g^{(-1)} \left( g(1)-g(1)+g(a) \right) \\
                &= g^{(-1)} \left( g(a) \right) \\
                &= a 
\end{align*}
    
\qed

\section{Exercise 3.3: Dual Triples}


\end{document}
